%!PN
%\documentclass[12pt,draft]{IEEEtran} %!PN
%\documentstyle[twocolumn]{IEEEtran}
%\documentstyle[12pt,twoside,draft]{IEEEtran}
%\documentstyle[9pt,twocolumn,technote,twoside]{IEEEtran}
%\documentclass[12pt]{article}
%\newtheorem{lemma}{Lemma}


\documentclass[twocolumn]{IEEEjar08_esp}
%%%%%%%%%%%%%%%%%%%%%%%%%%%%%%%%%%%%%%%%%%%%%%%%%%%%%%%%%%%%%%%%%%%%%%%%%%%%%%%%%%%%%%%%%%%%%%%%%%%%%%%%%%%%%%%%%%%%%%%%%%%%
\usepackage[latin1]{inputenc}
\usepackage{epsfig,graphics,graphicx}

%TCIDATA{Created=Tue Feb 26 18:39:06 2008}
%TCIDATA{LastRevised=Tue Feb 26 19:51:13 2008}
%TCIDATA{Language=American English}

\def\df{\; \stackrel \triangle = \;}
\def\BibTeX{{\rm B\kern-.05em{\sc i\kern-.025em b}\kern-.08em
    T\kern-.1667em\lower.7ex\hbox{E}\kern-.125emX}}
\setcounter{page}{1}
\input{tcilatex}
\begin{document}

\title{Preparaci�n de Trabajos para Conferencias patrocinadas por el IEEE}
\author{Nombre y apellido de los autores \\
%EndAName
Nombre de la Instituci�n y/o Empresa}
\maketitle

\begin{abstract}
El Resumen est� limitado a 100 palabras.
\end{abstract}

\section{Introducci�n}

El trabajo deber� presentarse en un archivo PDF y no deber� exceder las 6 p�%
ginas en tama�o A4. El formato ser� tipo IEEE en doble columna justificado. 

Los m�rgenes a tener en cuenta son: superior = 19 mm, inferior = 43 mm,
laterales = 13 mm. El ancho de la columna es de 88 mm. El espacio entre la
dos columnas es de 4 mm. La sangr�a de los p�rrafos es de 3,5 mm. 

La primera p�gina deber� incluir: t�tulo, nombre y apellido de cada autor,
nombre de la Instituci�n y/o Empresa a la que pertenecen si correspondiera,
su direcci�n de correo electr�nico y resumen (50 a 100 palabras) en el que
se mencionar�n brevemente los aspectos m�s relevantes del trabajo. 

\section{Consideracions Generales}

El uso de ''yo'', ''nosotros'', etc. deber� ser evitado.

Los t�tulos de las secciones deben estar en ''versales''. Numerar las
secciones con n�meros romanos y las subsecciones con letras. No numere
''AGRADECIMIENTOS'' ni ''REFERENCIAS''.

\subsection{Figuras y ecuaciones}

Usar la abreviatura ''Fig. 1'', incluso al comienzo de una oraci�n.

Numerar las ecuaciones sobre el margen derecho y consecutivamente. Citar la
ecuaci�n como ''(1)'' en lugar de ''Ec. (1)'' o ''ecuaci�n (1)'', excepto al
comienzo de una oraci�n. En ese caso se escribir�, por ejemplo, ''La ecuaci�%
n (1) es.''.

\subsection{Referencias}

Numerar las citas en forma consecutiva y entre corchetes. Citar simplemente
dicho n�mero, como en ''Los resultados fueron presentados en [1]'', a menos
que sea el comienzo de una oraci�n (''La referencia [1] presenta''). Seguir
el estilo que se ejemplifica a continuaci�n.

\begin{thebibliography}{9}
\bibitem{Win}  G. Eason, B. Noble, and I.N. Sneddon, ''On certain integrals
of Lipschitz-Hankel type involving products of Bessel functions,'' Phil.
Trans. Roy. Soc. London, vol. A247, pp. 529-551, April 1955.

\bibitem{Kal}  J. Clerk Maxwell, A Treatise on Electricity and Magnetism,
3rd ed., vol. 2. Oxford: Clarendon, 1892, pp.68-73.

\bibitem{Anderson}  I.S. Jacobs y C.P. Bean, ''Fine particles, thin films
and exchange anisotropy,'' in Magnetism, vol. III, G.T. Rado and H. Suhl,
Eds. New York: Academic, 1963, pp. 271-350.

\bibitem{Shaked}  K. Elissa, ''Title of paper if known,'' no puplicado.

\bibitem{Alhen}  R. Nicole, ''Title of paper with only first word
capitalized,'' J. Name Stand. Abbrev., en impresi�n.

\bibitem{Grimble1}  Y. Yorozu, M. Hirano, K. Oka, y Y. Tagawa, ''Electron
spectroscopy studies on magneto-optical media and plastic substrate
interface,'' IEEE Transl. J. Magn. Japan, vol. 2, pp. 740-741, August 1987
[Digests 9th Annual Conf. Magnetics Japan, p. 301, 1982].

\bibitem{Roberts}  M. Young, The Technical Writer's Handbook. Mill Valley,
CA: University Science, 1989.
\end{thebibliography}

\end{document}
